\documentclass[a4paper,12pt,twoside,openany]{report}
%
% Wzorzec pracy dyplomowej
% J. Starzynski (jstar@iem.pw.edu.pl) na podstawie pracy dyplomowej
% mgr. inż. Błażeja Wincenciaka
% Wersja 0.1 - 8 października 2016
%
\usepackage{polski}
\usepackage{helvet}
\usepackage[T1]{fontenc}
\usepackage{anyfontsize}
\usepackage[utf8]{inputenc}
\usepackage[pdftex]{graphicx}
\usepackage{tabularx}
\usepackage{array}
\usepackage[polish]{babel}
\usepackage{subfigure}
\usepackage{amsfonts}
\usepackage{verbatim}
\usepackage{indentfirst}
\usepackage[pdftex]{hyperref}


% rozmaite polecenia pomocnicze
% gdzie rysunki?
\newcommand{\ImgPath}{.}

% oznaczenie rzeczy do zrobienia/poprawienia
\newcommand{\TODO}{\textbf{TODO}}


% wyroznienie slow kluczowych
\newcommand{\tech}{\texttt}

% na oprawe (1.0cm - 0.7cm)*2 = 0.6cm
% na oprawe (1.1cm - 0.7cm)*2 = 0.8cm
%  oddsidemargin lewy margines na nieparzystych stronach
% evensidemargin lewy margines na parzystych stronach
\def\oprawa{1.05cm}
\addtolength{\oddsidemargin}{\oprawa}
\addtolength{\evensidemargin}{-\oprawa}

% table span multirows
\usepackage{multirow}
\usepackage{enumitem}	% enumitem.pdf
\setlist{listparindent=\parindent, parsep=\parskip} % potrzebuje enumitem

%%%%%%%%%%%%%%% Dodatkowe Pakiety %%%%%%%%%%%%%%%%%
\usepackage{prmag2017}   % definiuje komendy opieku,nrindeksu, rodzaj pracy, ...


%%%%%%%%%%%%%%% Strona Tytułowa %%%%%%%%%%%%%%%%%
% To trzeba wypelnic swoimi danymi
\title{Aplikacja do zarządzania publikacjami naukowymi z automatyczną analizą PDF}

% autor
\author{Piotr Jeleniewicz}
\nrindeksu{291072}



\opiekun{dr inż. Bartosz Chaber}
%\konsultant{prof. Dzielny Konsultant}  % opcjonalnie
\terminwykonania{1 lutego 2021} % data na oświadczeniu o samodzielności
\rok{2021}


% Podziekowanie - opcjonalne
\podziekowania{\noindent
{\Large Podziękowania}
\bigskip



\bigskip

{\raggedleft
Piotr Jeleniewicz

}

}

% To sa domyslne wartosci
% - mozna je zmienic, jesli praca jest pisana gdzie indziej niz w ZETiIS
% - mozna je wyrzucic jesli praca jest pisana w ZETiIS
%\miasto{Warszawa}
%\uczelnia{POLITECHNIKA WARSZAWSKA}
%\wydzial{WYDZIAŁ ELEKTRYCZNY}
%\instytut{INSTYTUT ELEKTROTECHNIKI TEORETYCZNEJ\linebreak[1] I~SYSTEMÓW INFORMACYJNO-POMIAROWYCH}
% \zaklad{ZAKŁAD ELEKTROTECHNIKI TEORETYCZNEJ\linebreak[1] I~INFORMATYKI STOSOWANEJ}
%\kierunekstudiow{INFORMATYKA}

% domyslnie praca jest inzynierska, ale po odkomentowaniu ponizszej linii zrobi sie magisterska
%\pracamagisterska
%%% koniec od P.W

\opinie{%
  \newpage
\begin{center}
 {\large\bf  Opinia} \\
o pracy dyplomowej magisterskiej wykonanej przez dyplomanta\\
{\bf Zdolnego Studenta i Pracowitego Kolegę} \\
 Wydział Elektryczny, kierunek Informatyka,  Politechnika Warszawska\\
Temat pracy\\
\textit{\bf
TYTUŁ PRACY DYPLOMOWEJ
}\\
\end{center}
\medskip
\noindent
Promotor: {\bf dr inż. Miły Opiekun}\\
Ocena pracy dyplomowej: {\bf bardzo dobry}

\medskip

\centerline{\bf Treść opinii}
   Celem pracy dyplomowej panów dolnego Studenta i Pracowitego Kolegi  było
opracowanie systemu pozwalającego symulować  i opartego o oprogramowanie o
otwartych źródłach (ang. Open Source). Jak piszą Dyplomanci, starali się opracować
system, który łatwo będzie dostosować do zmieniających się dynamicznie wymagań,
będzie miał niewielkie wymagania sprzętowe i umożliwiał dalszą łatwą rozbudowę oraz
dostosowanie go do potrzeb.
Przedstawiona do recenzji praca składa się z krótkiego wstępu jasno i
wyczerpująco opisującego oraz uzasadniającego cel pracy, trzech rozdziałów (2-4)
zawierających opis istniejących podobnych
rozwiązań, komponentów rozpatrywanychjako kandydaci do
tworzonego systemu i wreszcie zagadnień wydajności wirtualnych
rozwiązań. Piąty rozdział to opis przygotowanego przez
Dyplomantów środowiska obejmujący opis konfiguracji
środowiska oraz przykładowe ćwiczenia laboratoryjne. Ostatni
rozdział pracy to opis możliwości dalszego
rozwoju projektu. W ramach przygotowania pracy Dyplomanci zebrali i przedstawili w
bardzo przejrzysty sposób duży zasób informacji, co świadczy o dobrej orientacji
w nowoczesnej i ciągle intensywnie rozwijanej tematyce stanowiącej
zakres pracy i o umiejętności przejrzystego przedstawienia tych
wyników. Praca zawiera dwa dodatki, z których pierwszy obejmuje wyniki
eksperymentów i badań nad wydajnością, a drugi to źródła
skryptów budujących środowisko.

 Dyplomanci dość
dobrze zrealizowali postawione przed nimi zadanie,
wykazali się więc umiejętnością zastosowania w praktyce wiedzy
przedstawionej w rozdziałach 2-4.  Uważam, że cele postawione w założeniach pracy zostały pomyślnie
zrealizowane. Proponuję ocenę bardzo dobrą (5).

\vskip 1cm
{
\raggedleft
(data, podpis)\kern1cm

}
  \newpage
  \newpage
\begin{center}
 {\large\bf  Recenzja } \\
pracy dyplomowej magisterskiej wykonanej przez dyplomanta\\
{\bf Zdolnego Studenta i Pracowitego Kolegę} \\
 Wydział Elektryczny, kierunek Informatyka,  Politechnika Warszawska\\
Temat pracy\\
\textit{\bf
TYTUŁ PRACY DYPLOMOWEJ
}\\
\end{center}
\medskip
\noindent
Recenzent: {\bf prof. nzw. dr hab. inż. Jan Surowy}\\
Ocena pracy dyplomowej: {\bf bardzo dobry}
\medskip


\centerline{\bf Treść recenzji}
   

\vskip 1cm
{
\raggedleft
(data, podpis)\kern1cm

}
}

\streszczenia{
  \newpage
\begin{center}
\large \bf

APLIKACJA DO ZARZĄDZANIA PUBLIKACJAMI NAUKOWYMI Z AUTOMATYCZNĄ ANALIZĄ PDF
\end{center}

\section*{Streszczenie}
Celem pracy jest przygotowanie systemu do zarządzania publikacjami wraz z automatyczną analizą plików PDF, której celem jest uzyskanie z tego pliku informacji dotyczących między innymi tytułu oraz autorów publikacji. System wykorzystuje architekturę klient-serwer, gdzie do odpowiedzialności serwera należy przetwarzanie i przechowywanie danych dotyczących publikacji, a rolą klienta będzie natomiast prezentacja informacji pochodzących z serwera oraz interakcja z użytkownikiem. Efektem prac są dwie aplikacje:
kliencka, która została przygotowana dla systemu Android w języku Kotlin oraz serwerowa, która działa w oparciu o środowisko Node.js. W tej pracy zostały zawarte szczegóły dotyczące implementacji poszczególnych funkcji aplikacji zarówno po stronie serwera oraz klienta, a także testy demonstrujące poprawność pracy całego systemu. 



\bigskip
{\noindent\bf Słowa kluczowe:} zarządzanie publikacjami, aplikacja mobilna, usługa sieciowa, Node.js, Android, Kotlin

\vskip 1cm


\begin{center}
\large \bf
\uppercase{Reference manager with automatic PDF analysis}
\end{center}

\section*{Abstract}
The goal of this thesis is to create a system to prepare publication management system with automatic analysis of PDF files, which purpose is to read from PDF file informations like title or authors of publication. The system uses a client-server architecture, where the server responsibilities are processing and storing publication's data and the client's role is to present information from the server and interact with the user. The result of the work are two applications,
client, which was prepared for Android in Kotlin, and server one, which works in the Node.js environment. This thesis contains details of the implementation of individual system functions, as well as tests demonstrating that system works correctly. 

\bigskip
{\noindent\bf Keywords:} publication managment, moblie application, 
network service, Node.js, Android, Kotlin

\vfill
}

\begin{document}
\maketitle

%-----------------
% Wstęp
%-----------------
\chapter{Wstęp}
Przy pisaniu artykułów naukowych często sięga się po efekty prac przedstawione w innych publikacjach naukowych. Wraz z pisaniem coraz większej liczby tego typu dokumentów, ilość wykorzystywanych w nich pozycji bibliograficznych może zacząć znacznie wzrastać co może utrudnić odnajdowanie potrzebnych publikacji w stale rozszerzającym się ich zbiorze. 

Dlatego też ta praca będzie przedstawiała efekty prac nad aplikacją do zarządzania publikacjami naukowymi z automatyczną analizą plików PDF, której głównym celem jest ułatwienie procesu zarządzania publikacjami naukowymi, które przechowywane są w formie plików PDF. System będzie składał się aplikacji klienckiej przygotowanej dla systemu Android oraz aplikacji serwerowej działającej w kontenerze Dockera.
\section{Założenia projektowe}

\hspace{10pt} 	System powstający w ramach pracy inżynierskiej będzie opierał się o następujące założenia: 	
\begin{enumerate}
	\item Aplikacja serwerowa będzie napisana przy użyciu Node.js oraz języka TypeScript.
	\item W celu ułatwienia konfiguracji środowiska deweloperskiego jaki produkcyjnego, baza danych wraz z aplikacją serwerową będą uruchamiane w sposób skonteneryzowany przy wykorzystaniu technologii Docker. 
	\item Aplikacja kliencka będzie przeznaczona na system Android oraz do jej napisania wykorzystany zostanie język Kotlin.
	\item W celu korzystania z aplikacji użytkownik będzie musiał utworzyć konto w systemie.
	\item Do uzyskania informacji z pliku PDF dotyczących publikacji wykorzystane zostanie API dostępne pod adresem: \newline \url{https://api.crossref.org/} 
	\item Aplikacje będą tworzone w sposób modułowy, umożliwiając stosunkowo łatwą możliwość rozbudowywania funkcjonalności.W
	\item Podczas implementacji obu aplikacji w miarę możliwości stosowane będą najlepsze praktyki programistyczne.
\end{enumerate}

W systemie będą dostępne następujące funkcjonalności:		
\begin{enumerate}
	\item Wyświetlanie listy publikacji.
	\item Wyświetlanie opisu publikacji naukowych.
	\item Tworzenie nowej publikacji w oparciu o metadane oraz numer DOI, jeśli są dostępne w dodawanym pliku PDF.
	\item Edycja publikacji.
	\item Pobranie pliku PDF powiązanego z daną publikacją.
\end{enumerate}
%-----------------
% \
%-----------------
\chapter{Przedstawienie wykorzystywanych technologii}
Przed rozpoczęciem implementacji projektu przeprowadzona została szczegółowa analiza dostępnych technologii zarówno w kontekście aplikacji serwerowej jak również klienckiej. W trakcie jej trwania pod uwagę brane były przede wszystkim aspekty dotyczące specyfik danych technologii takich jak wydajność czy też sugerowane przeznaczenie poszczególnych rozwiązań, ale także kwestie dotyczące osobistych preferencji odnoszących się do danych technologii. 
  %-----------------
  % Historia
  %-----------------
\section{Aplikacja serwerowa}
W obecnych czasach ilość technologii pozwalających na pisanie aplikacji serwerowych jest ogromna. Podjęcie jednoznacznego wyboru dotyczącego, której z nich należy użyć do danego zadania jest niemalże niemożliwe, jednakże każda z nich cechuje się swoimi indywidualnymi cechami, które pozwalają w pewnym stopniu ocenić, które z dostępnych narzędzi będzie odpowiednie do rozwiązania wybranego problemu. W przypadku aplikacji serwerowej, tworzonej w ramach tej pracy, głównymi cechami branymi pod uwagę była wydajność danej technologii oraz ilość kodu wymagana do poprawnego i bezpiecznego działania aplikacji. Po przeanalizowaniu dostępnych rozwiązań, aplikacja zostanie napisana w środowisku Node.js, przy użyciu języka TypeScript. Do przechowywania danych wykorzystana zostanie baz danych MongoDB.

\subsection{Node.js}
Node.js jest środowiskiem uruchomieniowym bazującym na koncepcie \textit{,,JavaScript~everywhere''}, który unifikuje język używany podczas procesu tworzenia aplikacji webowej, wykorzystując JavaScript zarówno po stronie klienta jaki i serwera. W związku z tym, że w ramach projektu aplikacja kliencka będzie stworzona w formie aplikacji mobilnej na system Android, koncept ten nie zostanie spełniony. jednakże przy tworzeniu hipotetycznej aplikacji klienckiej działającej w przeglądarkach internetowych wykorzystywanie jednego języka, jest  sytuacją ułatwiającej rozwój i utrzymania takiej aplikacji.

Do niewątpliwych zalet środowiska Node.js należy ogromna ilość bibliotek dostępnych do wykorzystania w aplikacji. Dzięki temu wiele problemów może zostać rozwiązanych przy użyciu gotowych komponentów, co znacznie wpływa na zmniejszenie ilości kodu, który jest wymagany do rozwiązania konkretnego problemu. Z punktu widzenia aplikacji serwerowej tworzonej w ramach tego projektu, niezwykle istotna okazała się dostępność biblioteki umożliwiającej przetwarzanie i odczyt danych z plików \textit{pdf}. 

Inną ważną cechą środowiska Node.js, w odróżnieniu od środowisk wykorzystujących  język Python, jest bardzo dobra obsługa standardu JSON, która pozwala na używanie obiektów zgodnych ze standardem JSON niemal jako standardowych obiektów języka JavaScript, co znacznie wpływa na ułatwienie komunikacji pomiędzy aplikacją serwerową a kliencką. 

\subsection{TypeScript}
Językiem który zostanie wykorzystany do napisania aplikacji serwerowej będzie TypeScipt. Jest on nadzbiorem języka JavaScript, utworzonym i utrzymywanym przez firmę Microsoft. Zasadniczą różnicą pomiędzy tymi języki jest typowanie.

Typowanie można podzielić między innymi na następujące dwa rodzaje:
\begin{itemize}
	\item Typowanie statyczne - zmienne mają ustalany typ w momencie deklaracji. Wykorzystywane w języku TypeScript
	\item Typowanie dynamiczne -  typ zmiennej wynika wynika z wartości jaka jest w niej przechowywana. Wykorzystywane w języku JavaScript.
\end{itemize}

Pomimo że typowanie dynamiczne pozwala na większą elastyczność przy wykorzystaniu zmiennych, typowanie statyczne pozwala na wykrycie części błędów w momencie kompilacji, bądź też transpilacji kodu.

W celu uruchomienia kodu napisanego w języku TypeScript, musi najpierw zostać dokonana jego transpilacja do języka JavaScript. Dopiero ten kod jest realnie wykonywany w środowisku uruchomieniowym.

\subsection{MongoDB}

\begin{thebibliography}{99}
\addcontentsline{toc}{chapter}{Bibliografia}
\bibitem{LOKI2}{daemon9, ,,LOKI2'', Phrack Magazine, Issue 51. 
\url{http://phrack.org}}
\bibitem{RWWWS}{van Hauser, Reverse WWW Shell,  THC, The Hacker's 
Choice.\newline \url{www.thc.org}}


\end{thebibliography}

\zakonczenie  % wklejenie recenzji i opinii

\end{document}
%+++ END +++
