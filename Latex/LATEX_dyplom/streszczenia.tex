\newpage
\begin{center}
\large \bf

APLIKACJA DO ZARZĄDZANIA PUBLIKACJAMI NAUKOWYMI Z AUTOMATYCZNĄ ANALIZĄ PDF
\end{center}

\section*{Streszczenie}
Celem pracy jest przygotowanie systemu do zarządzania publikacjami wraz z automatyczną analizą plików PDF, której celem jest uzyskanie z tego pliku informacji dotyczących między innymi tytułu oraz autorów publikacji. System wykorzystuje architekturę klient-serwer, gdzie do odpowiedzialności serwera należy przetwarzanie i przechowywanie danych dotyczących publikacji, a rolą klienta będzie natomiast prezentacja informacji pochodzących z serwera oraz interakcja z użytkownikiem. Efektem prac są dwie aplikacje:
kliencka, która została przygotowana dla systemu Android w języku Kotlin oraz serwerowa, która działa w oparciu o środowisko Node.js. W tej pracy zostały zawarte szczegóły dotyczące implementacji poszczególnych funkcji aplikacji zarówno po stronie serwera oraz klienta, a także testy demonstrujące poprawność pracy całego systemu. 



\bigskip
{\noindent\bf Słowa kluczowe:} zarządzanie publikacjami, aplikacja mobilna, usługa sieciowa, Node.js, Android, Kotlin

\vskip 1cm


\begin{center}
\large \bf
\uppercase{Reference manager with automatic PDF analysis}
\end{center}

\section*{Abstract}
The goal of this thesis is to create a system to prepare publication management system with automatic analysis of PDF files, which purpose is to read from PDF file informations like title or authors of publication. The system uses a client-server architecture, where the server responsibilities are processing and storing publication's data and the client's role is to present information from the server and interact with the user. The result of the work are two applications,
client, which was prepared for Android in Kotlin, and server one, which works in the Node.js environment. This thesis contains details of the implementation of individual system functions, as well as tests demonstrating that system works correctly. 

\bigskip
{\noindent\bf Keywords:} publication managment, moblie application, 
network service, Node.js, Android, Kotlin

\vfill